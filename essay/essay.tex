\documentclass{article}
\usepackage{ctex}
\usepackage{import}
\usepackage{hyperref}	% 用于交叉引用
\usepackage{setspace}	% 用于设置行间距
\usepackage{listings}	% 用于代码高亮
\usepackage{xcolor}		% 用于处理颜色
\usepackage{ulem}		% 用于各种线
\usepackage{amsmath}	% 用于数学公式(如 \begin{align})
\usepackage{amsthm}		% 用于数学版式(如 \newtheorem{cmd}{caption})
\usepackage{booktabs}	% 用于表格画线
\usepackage{graphicx}	% 用于插入图片
\usepackage{minted}
\usepackage{amssymb}

\usepackage{algorithm}  
\usepackage{algorithmicx}  
\usepackage{algpseudocode}  
\usepackage{amsmath}  % 用于伪代码

\floatname{algorithm}{算法}
\renewcommand{\algorithmicrequire}{\textbf{输入:}}
\renewcommand{\algorithmicensure}{\textbf{输出:}}


\usepackage[top = 0.8in, bottom = 0.8in, left = 0.8in, right = 0.8in]{geometry} %设置页边距


\title{研究性学习结题报告书}
\author{}
\begin{document}
\maketitle
\begin{itemize}
	\item[\textbf{课题名称}] 用信息方法研究遗传学问题 \centering
	\item[\textbf{课题负责人}] 杨景云
	\item[\textbf{课题成员}] blablabla
	\item[\textbf{指导教师}] 李丽华老师
	\item[\textbf{所在班级}] 高二(9)班
\end{itemize}
\newpage
\section{约定}
	\begin{itemize}
		\item[\textbf{真值运算符}] 若 $[]$ 内表达式为真,则是 $1$,否则是 $0$。
	\end{itemize}
\section{定义}

\subsection{基因集合}

我们用 $\mathbb{G}$ 来表示基因集合。

对于只有显隐性的情况,基因集合由一系列大写字母和小写字母组成,大写字母表示显性,小写字母表示隐性。对于只有两对等位基因 $\texttt{A,B}$ 的情况,$\mathbb{G}=\{\texttt{A,B,a,b}\}$。

对于另一些更复杂的情况,拿喷瓜举例,基因集合可以写作 $\mathbb{G}=\{\texttt{g}^{-},\texttt{g}^{+},\texttt{G}\}$。

\subsection{对于集合元素的标号}

创建基因集合到 $\{1,2,\cdots |\mathbb{G}|\}$ 的映射 $f:\mathbb{G} \to \mathbb{Z}$。

基因的顺序就是标号的顺序。

容易发现其有逆运算 $f'$。

\subsection{集合到向量的转化}

一个集合 $S$ 可以转化为一个 $|S|$ 维向量 $v$,其中 $v_i=[f'(i) \in S]$。

若基因集合为 $\{\texttt{A,B}\}$,$\texttt{A}$ 标号为 $1$,$\texttt{B}$ 标号为 $2$,那么集合 $\{\texttt{A}\}$ 可以转化为 $(1,0)$。

\subsection{基因片段}

基因片段是一个向量。记基因片段组成的集合为 $\mathbb{P}$。

\subsubsection{配子基因片段}

我们用 $\vec G$ 来表示配子基因片段。

我们可以将一个具有 $k$ 个基因的配子用一个 $k$ 维有序向量 $\{a_i\}$ 表示,$a_i \in \mathbb{G}$。

\subsubsection{个体基因片段}

我们用 $\vec I$ 来表示个体基因片段。

我们可以将一个具有 $k$ 对等位基因的个体用一个 $k$ 维有序向量 $\{(l_i,r_i)\}$表示,$l_i,r_i \in \mathbb{G}$。

\subsection{基因片段的运算}

\subsubsection{加法运算 $+$}

对于 $L,R \in \mathbb{P}$,而且 $L,R$ 同为配子基因片段或个体基因片段,定义加法运算为两基因片段的 \textbf{有序} 拼接。

如 $(\texttt{A},\texttt{C}) + (\texttt{B}) = (\texttt{A},\texttt{B},\texttt{C})$。

\subsubsection{结合运算 $\oplus$}

对于 $L,R \in \mathbb{P}$,而且 $L,R$ 同为配子基因片段,而且长度相等,定义结合运算为按位有序结合:

$$(L \oplus R)_i=(\max(L_i,R_i),\min(L_i,R_i))$$

$\max,\min$ 为取序号较大/较小者。

如 $(\texttt{A},\texttt{b}) + (\texttt{a},\texttt{B})=(\texttt{(A,a)},\texttt{(B,b)})$。

\subsection{生成函数 (Generating function)} 

定义:

$$\textbf{A}=\sum_{i} a_i x^{i}$$

是序列 $\{a_i\}$ 的生成函数。

我们不关心 $x$ 的取值和级数是否收敛,把 $x$ 作为形式,只关心系数 $a_i$。

\subsection{基因片段生成函数}

定义:

$$\textbf{A}=\sum_{i \in \mathbb{P}} a_i x^{i}$$

是序列 $\{a_i\}$ 的基因片段生成函数。

\subsection{基因片段生成函数的系列运算}

\subsubsection{乘法运算 $\times$}

$$x^L \times x^R=x^{L+R}$$

\subsubsection{结合乘法运算 $\otimes$}

$$x^L \otimes x^R=x^{L \oplus R}$$

\subsection{基因片段生成函数的应用}

\textbf{求基因型为 $\texttt{AaBB}$ 的个体产生的配子数量比。}

构造生成函数:

$$\begin{aligned}
	\textbf{G} &= (\frac{1}{2} x^{\texttt{A}}+ \frac{1}{2}x^{\texttt{a}})(\frac{1}{2} x^{\texttt{B}}+ \frac{1}{2}x^{\texttt{B}}) \\
				&= \frac{1}{2} x^{\texttt{AB}} + \frac{1}{2} x^{\texttt{aB}}
\end{aligned}$$

即配子数量比为 $\texttt{AB} : \texttt{aB}=1:1$。

\textbf{求其自交后个体的基因型比例。}

构造生成函数:

$$\begin{aligned}
	\textbf{I} &= \textbf{G} \otimes \textbf{G} \\
				&= \frac{1}{4} x^{\texttt{AABB}} + \frac{1}{2} x^{\texttt{AaBB}} + \frac{1}{4} x^{\texttt{aaBB}}
\end{aligned}$$

即基因型数量比为 $\texttt{AABB} : \texttt{AaBB} : \texttt{aaBB}=1:2:1$。

\subsection{表现型集合}

定义 $\mathbb{E}$ 为表现型集合,一般地,$\mathbb{E}=\mathbb{G}$。

\subsection{表现型映射}

我们创建映射:$exp:\mathbb{G} \times \mathbb{G} \to \mathbb{E}$,对于一对等位基因 $l,r \in G$ 使得 $exp(l,r)$ 为这个个体的表现型。

比如 $exp(\texttt{A},\texttt{a})=\texttt{A}$,$exp(\texttt{a},\texttt{a})=\texttt{a}$。

\subsection{表现型映射的性质}

\begin{itemize}
	\item $exp(i,j)=exp(j,i)$。
	\item $exp(i,i)=i$。
\end{itemize}

\subsection{计算个体的表现型}

个体的表现型可以用一个 $k$ 维向量 $\vec E$ 表示,其中

$$\vec E_i=exp(\vec I_i)$$

\subsection{卷积}

给定环 $R$ 上的 $n$ 维向量 $\vec A=\{a_i\},\vec B=\{b_i\}$ 和下标运算 $\circ$,设 $C=\{c_i\}=A*B$,则满足:

\begin{equation}
	c_i=\sum_{j,k} [j \circ k=i] a_jb_k \label{(1)}
\end{equation}

称 $C$ 为 $A$ 和 $B$ 关于 $\circ$ 的离散卷积,以下简称卷积。

记 $C=A*_{\circ}B$,如果不引起混淆,简记为 $C=A*B$,其中 $*$ 为卷积算子。

若 $\circ = +$,就是我们熟悉的多项式乘法运算。

\subsection{卷积与生成函数运算的联系}

若满足运算 $x^L \times x^R = x^{L \circ R}$,那么生成函数 $\mathbf{F}=\sum f_i x^i$ 的乘法:

$$\textbf{H}=\textbf{F} \times \textbf{G}$$

和卷积 $\vec F=\{f_i\},\vec G=\{g_i\},\vec H=\vec F *_{\circ} \vec G=\{h_i\}$ 等价。

\section{只有显隐性情况群体自由交配的计算}

参考 2.9 中做法,我们需要分成两部分计算,第一部分是求配子生成函数 \textbf{G},第二部分是求基因片段生成函数 \textbf{I}。

\subsection{配子生成函数的求法}

将基因片段对应到一个二进制数,如 $\texttt{AB}=(11)_2=3,\texttt{aB}=(01)_2=1$。

\subsubsection{朴素求法}

模拟生成配子的过程,每次生成一个长度为 $k$ 的二进制数,若第 $i$ 位为 $0$,则选择第 $i$ 对等位基因的其中一个,否则选择另一个。

拿 $\texttt{AaBB}$ 举例:

\begin{table}[htbp]
	\centering
	\begin{tabular}{|c|c|}
		选择的二进制数 & 得到的配子 \\
		00 & \texttt{AB} \\
		01 & \texttt{AB} \\
		10 & \texttt{aB} \\
		11 & \texttt{aB} \\
	\end{tabular}
\end{table}

生成二进制数的时间复杂度(time complexity)为 $\mathcal O(2^k)$,而计算配子的时间复杂度为 $\mathcal O(k)$。

所以总时间复杂度是 $\mathcal O(k 2^k)$,对于 $n$ 个个体都计算一次,时间复杂度为 $\mathcal O(nk 2^k)$,是不能接受的。

\subsubsection{快速做法}

考虑维护配子出现次数函数 $f$,一开始为 $x^{\texttt{None}}$,考虑每次加入一对基因,$f$ 的变化。假设它变为 $f'$。

若加入的基因是一对显性基因,如 $\texttt{AA}$,那么 $f'(x \times 2 + 1)=2f(x)$。

若加入的基因是一个显性和一个隐形基因,如 $\texttt{Aa}$,那么 $f'(x \times 2 + 1)=f(x),f'(x \times 2)=f(x)$。

若加入的基因是一对隐性基因,如 $\texttt{aa}$,那么$f'(x \times 2)=2f(x)$。

加入 $k$ 等位基因,每次都 $\mathcal O(2^k)$ 计算,时间复杂度和上面没有区别,看似没有优化。

但是程序处理时,加入到第 $i$ 个等位基因时,可以只用考虑 $0 \sim 2^i$ 的函数值,总时间复杂度是 $\mathcal O(\sum_{i=1}^k 2^i)=\mathcal O(2^k)$,可以将一个 $k$ 优化掉。

对于 $n$ 个个体都计算一次,时间复杂度为 $\mathcal O(n2^k)$,比较快速。

\subsection{基因片段生成函数的求法}

我们想求出一个基因片段生成函数乘法的快速实现。

\subsubsection*{朴素做法}

考虑朴素地实现 $(1)$ 中的卷积,要枚举 $j,k$,通过下标运算 $\circ$ 计算出 $c_i$,时间复杂度是 $\mathcal O(4^k)$,是不能接受的。

\subsubsection*{优化的第一步}

我们发现 \textbf{对于只有显隐性情况的基因片段生成函数,可以转化为集合生成函数}。而且集合生成函数已经存在快速算法。

\subsubsection*{集合生成函数}

可以使用符号:

$$f=\sum_{S \subseteq U} f_S x^S$$

来表示一个集合生成函数。

这里我们定义算子 $\circ=\cup$,即:$x^L \times x^R=x^{L \cup R}$。

容易发现集合生成函数的乘法运算恰好为 \textbf{集合并卷积}。

\subsubsection*{基因片段生成函数到集合生成函数的转换}

定义全集 $U$ 是:$\{\texttt{A},\texttt{B},\cdots\}$。

我们将基因片段中的显性基因抽取出来,形成一个集合,如 $\texttt{ABc} \Rightarrow \{\texttt{A},\texttt{B}\}$。

这样发现集合并卷积刚好符合“显性基因克制隐形基因”的条件,因为只要某一位有对应的显性基因,那么个体就表现为显性,可以结合集合运算表来理解:

\begin{table}[htbp]
	\centering
	\begin{tabular}{|c|c|c|}
		\hline
		$\cup$ & $\{\texttt{A}\}$ & $\varnothing$ \\ \hline
		$\{\texttt{A}\}$ & $\{\texttt{A}\}$ & $\{\texttt{A}\}$\\ \hline
		$\varnothing$ & $\{\texttt{A}\}$ & $\varnothing$ \\ \hline
	\end{tabular}
\end{table}


\subsubsection*{集合生成函数的快速卷积算法:FWT}

仿照 FFT 的思路,\textbf{我们求出 $f$ 的一种变换 $\hat f$,使得 $f * g = h \Rightarrow \hat f_i \times \hat g_i = \hat h_i$},即将系数表示法转化为点值表示法。

我们给出关于集合并卷积的 FWT 运算,即快速莫比乌斯变换。

$$\hat f_S=\sum_{T \subseteq S} f_T$$

证明:

$$
\begin{aligned}
	\hat h_S &=\sum_{L} \sum_{R} [(L \cup R) \subseteq S] f_L g_R \\
		 &= \sum_{L} \sum_{R} [L \subseteq S][R \subseteq S] f_L g_R \\
		 &= \sum_{L} [L \subseteq S] f_L \sum_{R} [R \subseteq S] g_R \\
		 &= \hat f_S \hat g_S
\end{aligned}
$$

我们求出 $\hat h_S$ 后,当然需要将 $\hat h$ 转化为 $h$,于是需要反演运算:

$$f_S=\sum_{T \subseteq S} (-1)^{|S|-|T|}\hat f_T$$

可以用容斥简单证明。

\subsubsection*{朴素的变换和反演的实现}

枚举 $T$ 和 $S$,并且判断是否 $T \subseteq S$,时间复杂度 $\mathcal O(4^n)$,没有太大的变化。

\subsubsection*{经过优化的变换和反演的实现}

通过程序精细实现,能够以 $\mathcal O(2^{|S|})$ 的时间复杂度枚举 $S$ 的子集。

如果对于所有的 $S \subseteq U$,都这样枚举子集,时间复杂度为:

$$\mathcal O(\sum_{i=1}^k \binom{k}{i}2^i)= \mathcal O(3^k)$$

比上述做法稍有进步。

\subsubsection*{进一步优化的变换和反演的实现}

我们使用递推的思路,推导出 $\hat f_S$。

设 $\hat f_S^{(i)}=\sum_{T\subseteq S}[(S\setminus T)\subseteq\{1,\cdots,i\}]f_T$,$\hat f_S^{(n)}$ 即是目标序列。

首先有 $\hat f_S^{(0)}=f_S$,因为只有当 $S \setminus T$ 为空集时,才能属于空集。

对于所有 $i\notin S$ 的 $S$,满足 $\hat f_S^{(i)}=\hat f_S^{(i-1)},\hat f_{S\cup\{i\}}^{(i)}=\hat f_S^{(i-1)}+\hat f_{S\cup\{i\}}^{(i-1)}$。

我们解释一下两个式子。

$$\begin{aligned}
\hat f_S^{(i)} &= \sum_{T \subseteq S}[(S \setminus T)\subseteq\{1,\cdots,i\}]f_T\\
               &= \sum_{T \subseteq S}[(S \setminus T)\subseteq\{1,\cdots,i-1\}]f_T\\
               &= \hat f_S^{(i-1)}
\end{aligned}$$

这里我们发现 $i \notin (S \setminus T)$,所以可以直接把 $\{i\}$ 去掉,也是等价的。

$$
\begin{aligned}
\hat f_{S\cup\{i\}}^{(i)} &= \sum_{T \subseteq (S\cup\{i\})}[(S\cup\{i\}) \setminus T)\subseteq\{1,\cdots,i\}]f_T\\
               &= \sum_{T \subseteq (S\cup\{i\}) \text{ and } i \notin T}[((S\cup\{i\}) \setminus T)\subseteq\{1,\cdots,i\}]f_T+\sum_{T \subseteq (S\cup\{i\}) \text{ and } i \in T}[((S\cup\{i\}) \setminus T)\subseteq\{1,\cdots,i-1\}]f_T\\
               &= \sum_{T \subseteq S \text{ and } i \notin T}[(S \setminus T)\subseteq\{1,\cdots,i-1\}]f_T+\sum_{T \subseteq (S\cup\{i\}) \text{ and } i \in T}[((S\cup\{i\}) \setminus T)\subseteq\{1,\cdots,i-1\}]f_T\\
               &= \hat f_S^{(i-1)}+\hat f_{S \cup \{i\}}^{(i-1)}
\end{aligned}
$$

这样,我们 $\mathcal{O}(n2^n)$ 求出 $\hat f_S,\hat g_S$,按位乘,然后再反演回去即可。


\subsubsection*{快速莫比乌斯变换和反演的伪代码实现}

\begin{algorithm}
	\caption{快速莫比乌斯变换}
	\begin{algorithmic}[1]
		\Require 集合幂级数 $f$
		\Ensure $f$ 的莫比乌斯变换
		\Function {FastMobiusTransform}{$f$}
			\For{$i \gets 1 \textbf{ to } n$}
				\For{$\textbf{all }S \subseteq U \setminus \{i\}$}
					\State $f_{S \cup \{i\}} \gets f_{S \cup \{i\}} + f_{S}$
				\EndFor
			\EndFor
			\State \Return{$f$}
		\EndFunction
	\end{algorithmic}
\end{algorithm}

\begin{algorithm}
	\caption{快速莫比乌斯反演}
	\begin{algorithmic}[1]
		\Require 集合幂级数 $f$
		\Ensure $f$ 的莫比乌斯反演
		\Function {FastMobiusInversion}{$f$}
			\For{$i \gets 1 \textbf{ to } n$}
				\For{$\textbf{all }S \subseteq U \setminus \{i\}$}
					\State $f_{S \cup \{i\}} \gets f_{S \cup \{i\}} - f_{S}$
				\EndFor
			\EndFor
			\State \Return{$f$}
		\EndFunction
	\end{algorithmic}
\end{algorithm}

\section{只有显隐性情况群体自由交配的计算的推广}

\subsection{共显性问题}

有一种花卉,基因型为 $\texttt{AA}$ 时表现为红色,基因型为 $\texttt{Aa}$ 时表现为粉色,基因型为 $\texttt{aa}$ 时表现为白色。

我们将基因片段中的显性和隐性基因抽取出来,形成一个集合,如 $\texttt{ABc} \Rightarrow \{\texttt{A},\texttt{B},\texttt{c}\}$。

也可以理解为把一对等位基因拆成两位:

\begin{itemize}
	\item $\texttt{A} \Rightarrow 10$
	\item $\texttt{a} \Rightarrow 01$
\end{itemize}

容易发现这样做的时间复杂度为 $\mathcal O(2k \times 2^{2k})=\mathcal O(2k \times 4^k)$,和朴素做法差不多,是不可接受的。

\subsection{喷瓜问题}

喷瓜的性别由等位基因 $\texttt{g}^{-},\texttt{g}^{+},\texttt{G}$ 决定,其中:

\begin{table}[htbp]
	\centering
	\begin{tabular}{|c|c|c|c|}
		\hline
		$\oplus$ & $\texttt{g}^{-}$ & $\texttt{g}^{+}$ & $\texttt{G}$\\ \bottomrule
		$\texttt{g}^{-}$ & $\texttt{g}^{-}$ & $\texttt{g}^{+}$ & $\texttt{G}$\\ \hline
		$\texttt{g}^{+}$ & $\texttt{g}^{+}$ & $\texttt{g}^{+}$ & $\texttt{G}$\\ \hline
		$\texttt{G}$ & $\texttt{G}$ & $\texttt{G}$ & $\texttt{G}$\\ \hline
	\end{tabular}
\end{table}

\begin{algorithm}
	\caption{多维广义离散傅里叶变换}
	\begin{algorithmic}[1]
		\Require 幂级数 $f$,单位根 $w_k$,操作符 $opr$ 代表正变换还是逆变换。
		\Ensure $f$ 的傅里叶变换
		\Function {FourierTransform}{$f,w_k,opr$}
			\If $\ opr=1$
				\State $matrix_{i,j}=w^{(i-1)(j-1)}_k$
			\Else
				\State $matrix_{i,j}=\frac{1}{k}w^{-(i-1)(j-1)}_k$
			\EndIf
			\For{$i \gets 1 \textbf{ to } n$}
				\For{$\texttt{The k vectors satisfying }1 \cdots k\texttt{ on the i-th bit and the other bits are same.}$}
					\State $v \gets \texttt{the k vectors}$
					\For{$j \gets 1 \textbf{ to } k$}
						\State $g_j \gets f_{v_j}$
					\EndFor
					\State $g \gets g \times matrix$
					\For{$j \gets 1 \textbf{ to } k$}
						\State $f_{v_j} \gets g_j$
					\EndFor
				\EndFor
			\EndFor
			\State \Return{$f$}
		\EndFunction
	\end{algorithmic}
\end{algorithm}
\end{document}