\documentclass{article}
\usepackage{ctex}
\usepackage{import}
\usepackage{hyperref}	% 用于交叉引用
\usepackage{setspace}	% 用于设置行间距
\usepackage{listings}	% 用于代码高亮
\usepackage{xcolor}		% 用于处理颜色
\usepackage{ulem}		% 用于各种线
\usepackage{amsmath}	% 用于数学公式(如 \begin{align})
\usepackage{amsthm}		% 用于数学版式(如 \newtheorem{cmd}{caption})
\usepackage{booktabs}	% 用于表格画线
\usepackage{graphicx}	% 用于插入图片
%\usepackage{minted}
\usepackage{amssymb}
\usepackage{tabularx}


\usepackage{makecell}
\usepackage{boldline}

\usepackage{algorithm}  
\usepackage{algorithmicx}  
\usepackage{algpseudocode}  
\usepackage{amsmath}  % 用于伪代码

\floatname{algorithm}{算法}
\renewcommand{\algorithmicrequire}{\textbf{输入:}}
\renewcommand{\algorithmicensure}{\textbf{输出:}}


\makeatletter
\def\hlinewd#1{%
\noalign{\ifnum0=`}\fi\hrule \@height #1 %
\futurelet\reserved@a\@xhline}
\makeatother

\usepackage[top = 0.8in, bottom = 0.8in, left = 0.8in, right = 0.8in]{geometry} %设置页边距


\title{研究性学习结题报告书}
\author{}
\begin{document}
\maketitle
\begin{itemize}
	\item[\textbf{课题名称}] 用信息方法研究遗传学问题 \centering
	\item[\textbf{课题负责人}] 杨景云
	\item[\textbf{课题成员}] blablabla
	\item[\textbf{指导教师}] 李丽华老师
	\item[\textbf{所在班级}] 高二(9)班
\end{itemize}
\begin{abstract}
	由于缺少系统的方法,高中生物中有关基因遗传的计算问题一直是学习中的难点。本文从高中生物中的常见问题切入,建立一个计算遗传学中表现型比例,基因型比例和最优杂交方案的数学模型并对该模型用计算机方法进行求解。
	
	本文将数学中的生成函数,计算机科学中快速求解卷积的算法,以及遗传学中的基本定律结合。通过对基因片段二进制编码后,引入集合幂级数来表示基因片段出现次数,那么产生配子和配子结合的过程就是集合幂级数的不同形式卷积。进而用快速莫比乌斯变换和快速沃尔什变换在很低的算法时间复杂度内求解该问题。
	
	该模型得出了手工求解表现型和基因型比例的通用方法,以及能用计算机高效求解的算法。文章最后,根据此模型求解了实际应用中的若干个问题。
\end{abstract}
\newpage
\section{约定}

\textsl{真值运算符} \qquad 若 $[]$ 内表达式为真,则是 $1$,否则是 $0$。
		 
\section{定义}

%此处是否应有生物学中配子,基因型,表现型等等的定义? 

\subsection{基因集合}

我们用 $\mathbb{G}$ 来表示\textsl{基因集合}。

对于只有显隐性的情况,基因集合由一系列大写字母和小写字母组成,大写字母表示显性,小写字母表示隐性。对于只有两对等位基因 $\texttt{A,B}$ 的情况,$\mathbb{G}=\{\texttt{A,B,a,b}\}$。

对于另一些更复杂的情况,拿喷瓜举例,基因集合可以写作 $\mathbb{G}=\{\texttt{g}^{-},\texttt{g}^{+},\texttt{G}\}$。

\subsection{对于集合元素的标号}

创建基因集合到 $\{1,2,\cdots |\mathbb{G}|\}$ 的映射 $f:\mathbb{G} \to \mathbb{Z}$。

基因的顺序就是标号的顺序。

容易发现其有逆运算 $f'$。

\subsection{集合到向量的转化}

一个集合 $S$ 可以转化为一个 $|S|$ 维向量 $v$,其中 $v_i=[f'(i) \in S]$。

若基因集合为 $\{\texttt{A,B}\}$,$\texttt{A}$ 标号为 $1$,$\texttt{B}$ 标号为 $2$,那么集合 $\{\texttt{A}\}$ 可以转化为 $(1,0)$。

\subsection{基因片段}

\textsl{基因片段}是一个向量。记基因片段组成的集合为 $\mathbb{P}$。

\subsubsection*{配子基因片段}

我们用 $\vec G$ 来表示\textsl{配子基因片段}。

我们可以将一个具有 $k$ 个基因的配子用一个 $k$ 维向量 $\{a_i\}$ 表示,$a_i \in \mathbb{G}$。

\subsubsection*{个体基因片段}

我们用 $\vec I$ 来表示\textsl{个体基因片段}。

我们可以将一个具有 $k$ 对等位基因的个体用一个 $k$ 维向量 $\{(l_i,r_i)\}$表示,$l_i,r_i \in \mathbb{G}$。

\subsection{基因片段的运算}

\subsubsection{加法运算 $+$}

对于 $L,R \in \mathbb{P}$,而且 $L,R$ 同为配子基因片段或个体基因片段,定义\textsl{加法运算}为两基因片段的有序拼接。

如 $(\texttt{A},\texttt{C}) + (\texttt{B}) = (\texttt{A},\texttt{B},\texttt{C})$。

\subsubsection{结合运算 $\oplus$}

对于 $L,R \in \mathbb{P}$,而且 $L,R$ 同为配子基因片段,而且长度相等,定义\textsl{结合运算}为按位有序结合:

$$(L \oplus R)_i=(\max(L_i,R_i),\min(L_i,R_i))$$

$\max,\min$ 为取序号较大/较小者。排序可以根据生物中通用的表示方法来定义。

如 $(\texttt{A},\texttt{b}) + (\texttt{a},\texttt{B})=((\texttt{A},\texttt{a}),(\texttt{B},\texttt{b}))$。

\subsection{生成函数 (Generating function)} 

定义:

$$A=\sum_{i} a_i x^{i}$$

是序列 $\{a_i\}$ 的\textsl{生成函数}。

我们不关心 $x$ 的取值和级数是否收敛,把 $x$ 作为形式,只关心系数 $a_i$。

\subsection{基因片段生成函数}

定义:

$$A=\sum_{i \in \mathbb{P}} a_i x^{i}$$

是序列 $\{a_i\}$ 的\textsl{基因片段生成函数}。

\subsection{基因片段生成函数的系列运算}

\subsubsection*{乘法运算 $\times$}

$$x^L \times x^R=x^{L+R}$$

\subsubsection*{结合乘法运算 $\otimes$}

$$x^L \otimes x^R=x^{L \oplus R}$$

\subsection{基因片段生成函数的应用}

\textbf{求基因型为 $\texttt{AaBB}$ 的个体产生的配子数量比}

构造生成函数:

$$\begin{aligned}
	G &= (\frac{1}{2} x^{\texttt{A}}+ \frac{1}{2}x^{\texttt{a}})(\frac{1}{2} x^{\texttt{B}}+ \frac{1}{2}x^{\texttt{B}}) \\
				&= \frac{1}{2} x^{\texttt{AB}} + \frac{1}{2} x^{\texttt{aB}}
\end{aligned}$$

即配子数量比为 $\texttt{AB} : \texttt{aB}=1:1$。

\textbf{求其自交后个体的基因型比例}

构造生成函数:

$$\begin{aligned}
	I &= G \otimes G \\
		&= \frac{1}{4} x^{\texttt{AABB}} + \frac{1}{2} x^{\texttt{AaBB}} + \frac{1}{4} x^{\texttt{aaBB}}
\end{aligned}$$

即基因型数量比为 $\texttt{AABB} : \texttt{AaBB} : \texttt{aaBB}=1:2:1$。

\subsection{表现型集合}

定义 $\mathbb{E}$ 为表现型集合,一般地,$\mathbb{E}=\mathbb{G}$。

\subsection{表现型映射}

我们创建映射:$\operatorname{exp}:\mathbb{G} \times \mathbb{G} \to \mathbb{E}$,对于一对等位基因 $l,r \in G$ 使得 $\operatorname{exp}(l,r)$ 为这个个体的表现型。

%exp是不是不太好,容易和指数函数弄混

比如 $\operatorname{exp}(\texttt{A},\texttt{a})=\texttt{A}$,$\operatorname{exp}(\texttt{a},\texttt{a})=\texttt{a}$。

\subsection{表现型映射的性质}

\begin{itemize}
	\item $\operatorname{exp}(i,j)=\operatorname{exp}(j,i)$。
	\item $\operatorname{exp}(i,i)=i$。
\end{itemize}

\subsection{计算个体的表现型}

个体的表现型可以用一个 $k$ 维向量 $\vec E$ 表示,其中

$$\vec E_i=\operatorname{exp}(\vec I_i)$$

\subsection{卷积}

给定环 $R$ 上的 $n$ 维向量 $\vec A=\{a_i\},\vec B=\{b_i\}$ 和下标运算 $\circ$,设 $C=\{c_i\}=A*B$,则满足:

\begin{equation}
	c_i=\sum_{j,k} [j \circ k=i] a_jb_k \label{(1)}
\end{equation}

称 $C$ 为 $A$ 和 $B$ 关于 $\circ$ 的离散卷积,以下简称卷积。

记 $C=A*_{\circ}B$,如果不引起混淆,简记为 $C=A*B$,其中 $*$ 为卷积算子。

若 $\circ = +$,就是我们熟悉的多项式乘法运算。

\subsection{卷积与生成函数运算的联系}

若满足运算 $x^L \times x^R = x^{L \circ R}$,那么生成函数 $F=\sum f_i x^i$ 的乘法:

$$H=F \times G$$

和卷积 $\vec F=\{f_i\},\vec G=\{g_i\},\vec H=\vec F *_{\circ} \vec G=\{h_i\}$ 等价。

\section{只有显隐性情况群体自由交配的计算}

参考 2.9 中做法,我们分步计算。

\begin{enumerate}
	\item 对于第 $i$ 个个体,求配子生成函数 $G_i$。
	\item 计算 $G=\sum_{i=1}^n G_i$。 
	\item 计算 $I=G \otimes G$,
\end{enumerate}

\subsection{配子生成函数的求法}

将基因片段对应到一个二进制数,如 $\texttt{AB}=(11)_2=3,\texttt{aB}=(01)_2=1$。

\subsubsection*{朴素求法}

模拟生成配子的过程,每次生成一个长度为 $k$ 的二进制数,若第 $i$ 位为 $0$,则选择第 $i$ 对等位基因的其中一个,否则选择另一个。

拿 $\texttt{AaBB}$ 举例:

\begin{table}[htbp]
	\centering
	\caption{配子计算表}
	\begin{tabular}{|c|c|}
		选择的二进制数 & 得到的配子 \\
		00 & \texttt{AB} \\
		01 & \texttt{AB} \\
		10 & \texttt{aB} \\
		11 & \texttt{aB} \\
	\end{tabular}
\end{table}

生成二进制数的\textsl{时间复杂度}为 $\mathcal O(2^k)$,而计算配子的时间复杂度为 $\mathcal O(k)$。

所以总时间复杂度是 $\mathcal O(k 2^k)$,对于 $n$ 个个体都计算一次,时间复杂度为 $\mathcal O(nk 2^k)$,是不能接受的。

\subsubsection*{快速做法}

考虑维护配子出现次数函数 $f$,一开始为 $x^{\texttt{None}}$,考虑每次加入一对基因,$f$ 的变化。假设它变为 $f'$。

若加入的基因是一对显性基因,如 $\texttt{AA}$,那么 $f'(x \times 2 + 1)=2f(x)$。

若加入的基因是一个显性和一个隐形基因,如 $\texttt{Aa}$,那么 $f'(x \times 2 + 1)=f(x),f'(x \times 2)=f(x)$。

若加入的基因是一对隐性基因,如 $\texttt{aa}$,那么$f'(x \times 2)=2f(x)$。

加入 $k$ 等位基因,每次都 $\mathcal O(2^k)$ 计算,时间复杂度和上面没有区别,看似没有优化。

但是程序处理时,加入到第 $i$ 个等位基因时,可以只用考虑 $0 \sim 2^i$ 的函数值,总时间复杂度是 $\mathcal O(\sum_{i=1}^k 2^i)=\mathcal O(2^k)$,可以将一个 $k$ 优化掉。

对于 $n$ 个个体都计算一次,时间复杂度为 $\mathcal O(n2^k)$,比较快速。

\subsection{基因片段生成函数的求法}

我们想求出一个基因片段生成函数乘法的快速实现。

\subsubsection*{朴素做法}

考虑朴素地实现 $(1)$ 中的卷积,时间复杂度是 $\mathcal O(4^k)$,是不能接受的。

\subsubsection*{优化的第一步}

我们发现 \textbf{对于只有显隐性情况的基因片段生成函数,可以转化为集合生成函数}。而且集合生成函数已经存在快速算法。

\subsubsection*{集合生成函数}

可以使用符号:

$$f=\sum_{S \subseteq U} f_S x^S$$

来表示一个\textsl{集合生成函数}。

这里我们定义算子 $\circ=\cup$,即:$x^L \times x^R=x^{L \cup R}$。

容易发现集合生成函数的乘法运算恰好为\textsl{集合并卷积}。

\subsubsection*{基因片段生成函数到集合生成函数的转换}

定义全集 $U$ 是:$\{\texttt{A},\texttt{B},\cdots\}$。

我们将基因片段中的显性基因抽取出来,形成一个集合,如 $\texttt{ABc} \Rightarrow \{\texttt{A},\texttt{B}\}$。

这样发现集合并卷积刚好符合“显性基因克制隐形基因”的条件,因为只要某一位有对应的显性基因,那么个体就表现为显性,可以结合集合运算表来理解:

\begin{table}[htbp]
	\centering
	\caption{集合运算表}
	\begin{tabular}{cV{3}c|c}
		$\cup$ & $\{\texttt{A}\}$ & $\varnothing$ \\ \hlinewd{1pt}
		$\{\texttt{A}\}$ & $\{\texttt{A}\}$ & $\{\texttt{A}\}$\\ \hline
		$\varnothing$ & $\{\texttt{A}\}$ & $\varnothing$
	\end{tabular}
\end{table}


\subsubsection*{集合生成函数的快速卷积算法:FWT}

仿照 FFT 的思路,\textbf{我们求出 $f$ 的一种变换 $\hat f$,使得 $f * g = h \Rightarrow \hat f_i \times \hat g_i = \hat h_i$},即将系数表示法转化为点值表示法。

我们给出关于集合并卷积的 FWT 运算,即快速莫比乌斯变换。

$$\hat f_S=\sum_{T \subseteq S} f_T$$

证明:

$$
\begin{aligned}
	\hat h_S &=\sum_{L} \sum_{R} [(L \cup R) \subseteq S] f_L g_R \\
		 &= \sum_{L} \sum_{R} [L \subseteq S][R \subseteq S] f_L g_R \\
		 &= \sum_{L} [L \subseteq S] f_L \sum_{R} [R \subseteq S] g_R \\
		 &= \hat f_S \hat g_S
\end{aligned}
$$

我们求出 $\hat h_S$ 后,当然需要将 $\hat h$ 转化为 $h$,于是需要反演运算:

$$f_S=\sum_{T \subseteq S} (-1)^{|S|-|T|}\hat f_T$$

由容斥原理可以证明。

\subsubsection*{朴素的变换和反演的实现}

枚举 $T$ 和 $S$,并且判断是否 $T \subseteq S$,时间复杂度 $\mathcal O(4^k)$,没有太大的变化。

\subsubsection*{经过优化的变换和反演的实现}

通过程序精细实现,能够以 $\mathcal O(2^{|S|})$ 的时间复杂度枚举 $S$ 的子集。

如果对于所有的 $S \subseteq U$,都这样枚举子集 $T$,时间复杂度为:

$$\mathcal O(\sum_{i=0}^k \binom{k}{i}2^i)= \mathcal O(3^k)$$

比朴素做法稍有进步。

\subsubsection*{进一步优化的变换和反演的实现}

我们使用递推的思路,推导出 $\hat f_S$。

设 $\hat f_S^{(i)}=\sum_{T\subseteq S}[(S\setminus T)\subseteq\{1,\cdots,i\}]f_T$,$\hat f_S^{(n)}$ 即是目标序列。

首先有 $\hat f_S^{(0)}=f_S$,因为只有当 $S \setminus T$ 为空集时,才能属于空集。

对于所有 $i\notin S$ 的 $S$,满足 $\hat f_S^{(i)}=\hat f_S^{(i-1)},\hat f_{S\cup\{i\}}^{(i)}=\hat f_S^{(i-1)}+\hat f_{S\cup\{i\}}^{(i-1)}$。

我们解释一下两个式子。

$$\begin{aligned}
\hat f_S^{(i)} &= \sum_{T \subseteq S}[(S \setminus T)\subseteq\{1,\cdots,i\}]f_T\\
               &= \sum_{T \subseteq S}[(S \setminus T)\subseteq\{1,\cdots,i-1\}]f_T\\
               &= \hat f_S^{(i-1)}
\end{aligned}$$

这里我们发现 $i \notin (S \setminus T)$,所以可以直接把 $\{i\}$ 去掉,也是等价的。

$$
\begin{aligned}
\hat f_{S\cup\{i\}}^{(i)} &= \sum_{T \subseteq (S\cup\{i\})}[(S\cup\{i\}) \setminus T)\subseteq\{1,\cdots,i\}]f_T\\
               &= \sum_{T \subseteq (S\cup\{i\}) \text{ and } i \notin T}[((S\cup\{i\}) \setminus T)\subseteq\{1,\cdots,i\}]f_T+\sum_{T \subseteq (S\cup\{i\}) \text{ and } i \in T}[((S\cup\{i\}) \setminus T)\subseteq\{1,\cdots,i-1\}]f_T\\
               &= \sum_{T \subseteq S \text{ and } i \notin T}[(S \setminus T)\subseteq\{1,\cdots,i-1\}]f_T+\sum_{T \subseteq (S\cup\{i\}) \text{ and } i \in T}[((S\cup\{i\}) \setminus T)\subseteq\{1,\cdots,i-1\}]f_T\\
               &= \hat f_S^{(i-1)}+\hat f_{S \cup \{i\}}^{(i-1)}
\end{aligned}
$$

这样,我们 $\mathcal{O}(n2^n)$ 求出 $\hat f_S,\hat g_S$,按位乘,然后再反演回去即可。


\subsubsection*{快速莫比乌斯变换和反演的伪代码实现}

\begin{algorithm}
	\caption{快速莫比乌斯变换}
	\begin{algorithmic}[1]
		\Require 集合幂级数 $f$
		\Ensure $f$ 的莫比乌斯变换
		\Function {FastMobiusTransform}{$f$}
			\For{$i \gets 1 \textbf{ to } n$}
				\For{$\textbf{all }S \subseteq U \setminus \{i\}$}
					\State $f_{S \cup \{i\}} \gets f_{S \cup \{i\}} + f_{S}$
				\EndFor
			\EndFor
			\State \Return{$f$}
		\EndFunction
	\end{algorithmic}
\end{algorithm}

\begin{algorithm}
	\caption{快速莫比乌斯反演}
	\begin{algorithmic}[1]
		\Require 集合幂级数 $f$
		\Ensure $f$ 的莫比乌斯反演
		\Function {FastMobiusInversion}{$f$}
			\For{$i \gets 1 \textbf{ to } n$}
				\For{$\textbf{all }S \subseteq U \setminus \{i\}$}
					\State $f_{S \cup \{i\}} \gets f_{S \cup \{i\}} - f_{S}$
				\EndFor
			\EndFor
			\State \Return{$f$}
		\EndFunction
	\end{algorithmic}
\end{algorithm}

\newpage

\section{只有显隐性情况群体自由交配的计算的推广}

\subsection{共显性问题}

有的时候一对等位基因对应的不只有一对相对性状,而有更复杂的情况。

例如有一种花卉,基因型为 $\texttt{AA}$ 时表现为红色,基因型为 $\texttt{Aa}$ 时表现为粉色,基因型为 $\texttt{aa}$ 时表现为白色。

\begin{table}[htbp]
	\centering
	\caption{共显性表现型表}
	\begin{tabular}{cV{3}c|c|c}
		$\operatorname{exp}$ & $\texttt{A}$ & $\texttt{a}$ \\ \hlinewd{1pt}
		$\texttt{A}$ & $\texttt{A}$ & $$\texttt{Aa}$$ \\ \hline
		$\texttt{a}$ & $\texttt{Aa}$ & $$\texttt{a}$$
	\end{tabular}
\end{table}

我们将基因片段中的显性和隐性基因抽取出来,形成一个集合,如 $\texttt{ABc} \Rightarrow \{\texttt{A},\texttt{B},\texttt{c}\}$,对这样的集合作集合并卷积,也可以理解为把一对等位基因拆成两位,$\texttt{A} \Rightarrow 10$,$\texttt{a} \Rightarrow 01$。

容易发现这样做的时间复杂度为 $\mathcal O(2k \times 2^{2k})=\mathcal O(2k \times 4^k)$,和朴素做法差不多,是不可接受的。

\subsection{复等位基因问题}

一对相对形状由多个等位基因决定.

例如 喷瓜的性别由等位基因 $\texttt{g}^{-},\texttt{g}^{+},\texttt{G}$ 决定,其中:

\begin{table}[htbp]
	\centering
	\caption{喷瓜表现型表}
	\begin{tabular}{cV{3}c|c|c}
		$\operatorname{exp}$ & $\texttt{g}^{-}$ & $\texttt{g}^{+}$ & $\texttt{G}$\\ \hlinewd{1pt}
		$\texttt{g}^{-}$ & $\texttt{g}^{-}$ & $\texttt{g}^{+}$ & $\texttt{G}$\\ \hline
		$\texttt{g}^{+}$ & $\texttt{g}^{+}$ & $\texttt{g}^{+}$ & $\texttt{G}$\\ \hline
		$\texttt{G}$ & $\texttt{G}$ & $\texttt{G}$ & $\texttt{G}$
	\end{tabular}
\end{table}

容易发现,这些等位基因构成一个\textsl{偏序集},我们发现若 $\texttt{g}^{-} \le \texttt{g}^{+} \le \texttt{G}$,则 $\operatorname{exp}$ 运算对应 $\max$ 运算。

将 $\texttt{g}^{-},\texttt{g}^{+},\texttt{G}$ 编码成为 $00,01,10$,那么容易看出:

\begin{table}[htbp]
	\centering
	\caption{编码运算表}
	\begin{tabular}{cV{3}c|c|c}
		$\operatorname{or}$ & $00(\texttt{g}^{-})$ & $01(\texttt{g}^{+})$ & $10(\texttt{G})$\\ \hlinewd{1pt}
		$00(\texttt{g}^{-})$ & $00(\texttt{g}^{-})$ & $01(\texttt{g}^{+})$ & $10(\texttt{G})$\\ \hline
		$01(\texttt{g}^{+})$ & $01(\texttt{g}^{+})$ & $01(\texttt{g}^{+})$ & $11(\texttt{G})$\\ \hline
		$10(\texttt{G})$ & $10(\texttt{G})$ & $11(\texttt{G})$ & $10(\texttt{G})$
	\end{tabular}
\end{table}

发现 $11,10$ 都对应 $\texttt{G}$,而 $00$ 对应 $\texttt{g}^-$,$01$ 对应 $\texttt{g}^+$。我们在程序实现时最后一步处理一下即可。

容易发现这样做的时间复杂度还是 $\mathcal O(2k \times 2^{2k})=\mathcal O(2k \times 4^k)$,和朴素做法差不多,是不可接受的。

\newpage

\section{高维 FWT}

以上两个问题在只运用集合并卷积的情况下,都没有较低时间复杂度的算法,下面,我们引入高维 FWT,并且逐渐探寻 FWT 的一般式。

\subsection{定义}

定义生成函数 $F=\sum f_S x^S$,其中 $S$ 不再是一个集合,而是一个每维可以取 $0,\cdots ,k-1$ 的向量。

\subsection{每维取 $\max$ 的 FWT}

容易看出,当 $k=2$,而且:

$$\begin{aligned}0 \circ 0 = 0 \\ 1 \circ 0 = 1 \\ 0 \circ 1 = 1 \\ 1 \circ 1 = 1\end{aligned}$$

那么,这就对应了集合并卷积。

这里,我们不再讨论集合并卷积,而是考虑更加一般的形式,即 $\circ = \max$ 时的情形。

定义:

$$\hat f[x^S]=\sum[S \circ T=S] f_T$$

容易发现:

$$\hat h_S=\sum_{L}\sum_{R}[S \circ (L \circ R) = S] f_L \times g_R$$

由于:

$$\max(a,\max(b,c))=a \Leftrightarrow \max(a,b)=a \operatorname{and} \max(a,c)=a$$

有:

$$[(S \circ (L \circ R))=S]=[(S \circ L) =S][(S \circ R)=S]$$

得:

$$\begin{aligned} \hat h_S &= \sum_{L} \sum_{R} [(S \circ L) =S][(S \circ R)=S] f_L \times g_R \\ &= \sum_{L} [(S \circ L)=S] f_L \times \sum_{R} [(S \circ R)=S] g_R \\ &= \hat f_S \times \hat g_S\end{aligned}$$


那么,我们在 FWT 的 $k$ 个向量中,取前缀和即可,如果是反演的话,相邻做差即可。

\subsection{喷瓜问题的快速算法}

通过上述算法,将 $\texttt{g}^-$ 对应到 $0$,$\texttt{g}^+$ 对应到 $1$,$\texttt{G}$ 对应到 $2$,我们就可以解决上述的喷瓜问题,时间复杂度为 $\mathcal O(n \times 3^n)$。

\subsection{任意操作符的 FWT 问题}

容易发现,每次 FWT,都是在对其他位相同,而某一位分别为 $0, \cdots ,k-1$ 的 $k$ 个向量对应的下标做矩阵乘法。

如集合并卷积的矩阵:

$$\begin{bmatrix} 1 & 0 \\ 1 & 1 \end{bmatrix}$$

集合交卷积的矩阵:

$$\begin{bmatrix} 1 & 1 \\ 0 & 1 \end{bmatrix}$$

集合对称差卷积的矩阵:

$$\begin{bmatrix} 1 & 1 \\ 1 & -1 \end{bmatrix}$$

上述 $\max$ 卷积的矩阵:

$$\begin{bmatrix} 1& 0 & 0& \cdots & 0\\ 1& 1& 0& \cdots & 0\\ 1& 1 & 1& \cdots & 0\\ \vdots& \vdots& \vdots& \ddots& \vdots\\ 1& 1& 1 & \cdots & 1 \end{bmatrix}$$

而反演则是乘对应的逆矩阵。

我们设矩阵为:

$$\mathbf{M}=\begin{bmatrix} a_{0,0}&  a_{0,1} &  a_{0,2}& \cdots &  a_{0,k-1}\\  a_{1,0}&  a_{1,1}&  a_{1,2}& \cdots &  a_{1,k-1}\\  a_{2,0}&  a_{2,1} &  a_{2,2}& \cdots &  a_{2,k-1}\\ \vdots& \vdots& \vdots& \ddots& \vdots\\  a_{k-1,0}&  a_{k-1,1}&  a_{k-1,2} & \cdots &  a_{k-1,k-1} \end{bmatrix}$$

由于 FWT 按位独立,对于某一维分析,有:

$$\left(\begin{bmatrix}f_0 \\ f_1 \\ f_2 \\ \vdots \\ f_{k-1} \end{bmatrix}\times \mathbf{M} \right) \cdot \left(\begin{bmatrix}g_0 \\ g_1 \\ g_2 \\ \vdots \\ g_{k-1} \end{bmatrix}\times \mathbf{M}\right) = \begin{bmatrix}h_0 \\ h_1 \\ h_2 \\ \vdots \\ h_{k-1} \end{bmatrix}\times \mathbf{M}$$

其中 $\cdot$ 代表“按位乘”,即:

$$\begin{bmatrix}f_0 \\ f_1 \\ f_2 \\ \vdots \\ f_{k-1} \end{bmatrix} \cdot \begin{bmatrix}g_0 \\ g_1 \\ g_2 \\ \vdots \\ g_{k-1} \end{bmatrix} = \begin{bmatrix}f_0 \times g_0 \\ f_1 \times g_1 \\ f_2 \times g_2 \\ \vdots \\ f_{k-1} \times g_{k-1} \end{bmatrix}$$

由于 $h_{i}=\sum_{j,k} [j \circ k =i]f_{j} \times g_{k}$,枚举每个 $i$,对于每个 $f_j \times g_k$ 分析,容易列出方程:

$$a_{i,j} \times a_{i,k}=a_{i,j \circ k}$$

发现不管对于哪个 $i$,方程都是一样的,去掉 $i$,我们就只用解方程 $a_{j} \times a_{k}=a_{j \circ k}$。

如,当 $\circ$ 运算为取 $\operatorname{or}$ 的时候,有:

$$
\begin{cases}
	a_{0} \times a_{0}=a_{0} \\
	a_{1} \times a_{0}=a_{1} \\
	a_{0} \times a_{1}=a_{1} \\
	a_{1} \times a_{1}=a_{1} \\
\end{cases}
$$

我们解出两组解:

$$
\begin{aligned}
    \begin{cases}
        a_{0} =1 \\
        a_{1} =0 \\
	\end{cases}
	\qquad
    \begin{cases}
        a_{0} =1 \\
        a_{1} =1 \\
    \end{cases}
\end{aligned}
$$

于是可以这样安排我们的矩阵:

$$\mathbf{M_1}=\begin{bmatrix} 1 & 0 \\ 1 & 1 \end{bmatrix}$$

$$\mathbf{M_2}=\begin{bmatrix} 1 & 1 \\ 1 & 0 \end{bmatrix}$$

为什么不能这样这样安排:

$$\mathbf{M_3}=\begin{bmatrix} 1 & 0 \\ 1 & 0 \end{bmatrix}$$

$$\mathbf{M_4}=\begin{bmatrix} 1 & 1 \\ 1 & 1 \end{bmatrix}$$

原因是,这两个矩阵都没有对应的逆矩阵,求逆矩阵可以再列出一个方程,然后解出 $x_0,x_1$。(可以注意到求解方程的意义正好对应了 FWT 逆操作的意义)

$$
\begin{cases}
    a_{0,0}x_0 + a_{0,1}x_1 =b_0 \\
    a_{1,0}x_0 + a_{1,1}x_1 =b_1 \\
\end{cases}
$$

拿 $\mathbf{M_1}$ 举例,有:

$$
\begin{cases}
    x_0 = b_1 \\
    x_0 + x_1 =b_2 \\
\end{cases}
$$

那么显然:

$$
\begin{cases}
	x_0 = b_1 \\
	x_1 = b_2-b_1 \\
\end{cases}
$$

于是其逆矩阵就是:

$$\mathbf{M_1^{-1}}=\begin{bmatrix} 1 & 0 \\ -1 & 1 \end{bmatrix}$$

而对于 $\mathbf{M_3}$ 来说,有:

$$
\begin{cases}
	x_0 = b_1 \\
	x_0 = b_2 \\
\end{cases}
$$

显然不合法。于是不能使用 $\mathbf{M_3}$ 这个矩阵。

这样,我们解出矩阵 $\mathbf{M}$,然后求出逆矩阵 $\mathbf{M^{-1}}$,就可以解决任意操作符 $\circ$ 的 FWT 问题。

\subsection{$\circ$ 运算需要满足的性质}

由于:

$$a_{j} \times a_{k}=a_{j \circ k}$$

有:

$$
a_{j \circ k}=a_j \times a_k=a_k \times a_j=a_{k \circ j}
$$

$$
a_{j \circ (k \circ l)}=a_j \times a_{k \circ l}=a_j \times a_k \times a_l=a_{j \circ k} \times a_{l}=a_{(j \circ k)\circ l}
$$

于是 $\circ$ 运算必须满足交换律和结合律。

\subsection{不进位加法的 FWT}

我们定义不进位加法 $\oplus_p$ 运算,为:

$$
a \oplus_p b  =
\left\{
    \begin{array}{l}
        a+b \quad (0 \le a+b \le p-1) \\
        a+b-p \quad (p \le a+b \le 2p-2) \\
    \end{array}
\right.
$$

容易发现,其矩阵系数 $a$ 满足:

$$a_{i,j} \times a_{i,k}=a_{i,j \oplus_p k}$$

这里,我们发现,这组方程的特解即是:

$$a_{i,j} = \omega_{p}^j$$

因为单位根运算满足:

$$\omega_{p}^k = \omega_{p}^{k+p}$$

$$\omega_{p}^{i+j} = \omega_{p}^i \times \omega_{p}^j$$

进而发现,方程有 $p$ 组解,第 $i$ 组解(从 $0$ 开始编号)为:

$$a_{i,j} = \omega_{p}^{j \times i}$$

那么我们可以列出矩阵:

$$
\begin{bmatrix} 1& 1 & 1& \cdots & 1\\ 1& \omega_p^1& \omega_p^2& \cdots & \omega_p^{p - 1}\\ 1& \omega_p^2 & \omega_p^4& \cdots & w_p^{2(p - 1)}\\ \vdots& \vdots& \vdots& \ddots& \vdots\\ 1& \omega_p^{p - 1}& \omega_p^{2(p - 1)} & \cdots & \omega_p^{(p - 1)(p - 1)} \end{bmatrix}
$$

此矩阵就是\textsl{范德蒙德矩阵}。

我们不加证明地给出它的逆矩阵:

$$
\frac 1 p \begin{bmatrix} 1& 1 & 1& \cdots & 1\\ 1& \omega_p^{-1}& \omega_p^{-2}& \cdots & \omega_p^{-(p - 1)}\\ 1& \omega_p^{-2} & \omega_p^{-4}& \cdots & \omega_p^{-2(p - 1)}\\ \vdots& \vdots& \vdots& \ddots& \vdots\\ 1& \omega_p^{-(p - 1)}& \omega_p^{-2(p - 1)} & \cdots & \omega_p^{-(p - 1)(p - 1)} \end{bmatrix}
$$

这样,我们就可以完成模 $p$ 意义下的不进位加法卷积,此算法即\textsl{多维广义离散傅里叶变换}。

具体程序实现,我们可以算出 $\omega_p^1$ 即 $\cos \frac{p}{2\pi} + i\sin \frac{p}{2\pi}$,如果能用根号形式表示,即:$a+b\sqrt{x}i$,我们可以模拟复数 $a+b\sqrt{x}i$,其乘法为 $(a+b\sqrt{x}i)(c+d\sqrt{x}i)=(ac-xbd)+(ac+bd)\sqrt{x}i$,加法为 $(a+b\sqrt xi)+(c+d\sqrt xi)=(a+b)+(c+d)\sqrt xi$。

或者,更加通用地,我们将长度为 $p$ 的多项式环作为一种数据结构,假设是 $F=\sum_{i=0}^{p-1} \omega_p^i f_i$,有:$F \times \omega_p^k=F=\sum_{i=0}^{p-1} \omega_p^i f_{i\ominus k}$,$F \times G=\sum_{i=0}^{p-1} \sum_{j=0}^{p-1} w_p^{i \oplus j} f_i \times g_j$。

注意,如果对某个数 $mod$ 取模,若 $p$ 在 $mod$ 下没有对应的逆,则不能使用此算法。

多维广义离散傅里叶变换的代码实现:

\begin{algorithm}
	\caption{多维广义离散傅里叶变换}
	\begin{algorithmic}[1]
		\Require 幂级数 $f$,单位根 $w_p$,操作符 $opr$ 代表正变换还是逆变换。
		\Ensure $f$ 的傅里叶变换
		\Function {FourierTransform}{$f,w_p,opr$}
			\If $\ opr=1$
				\State $\textbf{M}_{i,j} \gets w^{(i-1)(j-1)}_p$
			\Else
				\State $\textbf{M}_{i,j} \gets \frac{1}{p}w^{-(i-1)(j-1)}_p$
			\EndIf
			\For{$i \gets 1 \textbf{ to } n$}
				\For{$\texttt{The p vectors satisfying }1 \cdots p\texttt{ on the i-th bit and the other bits are same.}$}
					\State $v \gets \texttt{the p vectors}$
					\For{$j \gets 1 \textbf{ to } p$}
						\State $g_j \gets f_{v_j}$
					\EndFor
					\State $g \gets g \times \textbf{M}$
					\For{$j \gets 1 \textbf{ to } p$}
						\State $f_{v_j} \gets g_j$
					\EndFor
				\EndFor
			\EndFor
			\State \Return{$f$}
		\EndFunction
	\end{algorithmic}
\end{algorithm}

\subsection{共显性问题的快速算法}

如果将 $\texttt{A}$ 对应到 $1$,$\texttt{a}$ 对应到 $0$,容易发现:

\begin{table}[htbp]
	\centering
	\caption{编码运算表}
	\begin{tabular}{cV{3}c|c}
		$+$ & $1(\texttt{A})$ & $0(\texttt{a})$\\ \hlinewd{1pt}
		$1(\texttt{A})$ & $2(\texttt{A})$ & $1(\texttt{Aa})$\\ \hline
		$0(\texttt{a})$ & $1(\texttt{Aa})$ & $0(\texttt{a})$\\
	\end{tabular}
\end{table}

我们发现 $\texttt{A}$ 对应 $2$,$\texttt{Aa}$ 对应 $1$,$\texttt{a}$ 对应 $0$。

只要使用三次单位根 $\omega_3=\cos 120^\circ + \sin 120^\circ \rm i$,即可轻松解决此问题,时间复杂度是 $\mathcal O(n \times 3^n)$。

\end{document}