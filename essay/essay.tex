\documentclass{article}
\usepackage{ctex}
\usepackage{import}
\usepackage{hyperref}	% 用于交叉引用
\usepackage{setspace}	% 用于设置行间距
\usepackage{listings}	% 用于代码高亮
\usepackage{xcolor}		% 用于处理颜色
\usepackage{ulem}		% 用于各种线
\usepackage{amsmath}	% 用于数学公式(如 \begin{align})
\usepackage{amsthm}		% 用于数学版式(如 \newtheorem{cmd}{caption})
\usepackage{booktabs}	% 用于表格画线
\usepackage{graphicx}	% 用于插入图片
\usepackage{minted}
\usepackage{amssymb}

\usepackage{algorithm}  
\usepackage{algorithmicx}  
\usepackage{algpseudocode}  
\usepackage{amsmath}  % 用于伪代码

\floatname{algorithm}{算法}
\renewcommand{\algorithmicrequire}{\textbf{输入:}}
\renewcommand{\algorithmicensure}{\textbf{输出:}}


\usepackage[top = 0.8in, bottom = 0.8in, left = 0.8in, right = 0.8in]{geometry} %设置页边距


\title{研究性学习结题报告书}
\author{}
\begin{document}
	\maketitle
	\begin{itemize}
		\item[\textbf{课题名称}] 用信息方法研究遗传学问题 \centering
		\item[\textbf{课题负责人}] 杨景云
		\item[\textbf{课题成员}] blablabla
		\item[\textbf{指导教师}] 李丽华老师
		\item[\textbf{所在班级}] 高二(9)班
	\end{itemize}
	\newpage
	\section{约定}
		\begin{itemize}
			\item[\textbf{真值运算符}] 若 $[]$ 内表达式为真,则是 $1$,否则是 $0$。
		\end{itemize}
	\section{定义}
	
	\subsection{基因集合}

	我们用 $\mathbb{G}$ 来表示基因集合。

	对于只有显隐性的情况,基因集合由一系列大写字母和小写字母组成,大写字母表示显性,小写字母表示隐性。对于只有两对等位基因 $\texttt{A,B}$ 的情况,$\mathbb{G}=\{\texttt{A,B,a,b}\}$。

	对于另一些更复杂的情况,拿喷瓜举例,基因集合可以写作 $\mathbb{G}=\{\texttt{g}^{-},\texttt{g}^{+},\texttt{G}\}$。
	
	\subsection{对于集合元素的标号}

	创建基因集合到 $\{1,2,\cdots |\mathbb{G}|\}$ 的映射 $f:\mathbb{G} \to \mathbb{Z}$。

	基因的顺序就是标号的顺序。
	
	容易发现其有逆运算 $f'$。

	\subsection{集合到向量的转化}

	一个集合 $S$ 可以转化为一个 $|S|$ 维向量 $v$,其中 $v_i=[f'(i) \in S]$。

	若基因集合为 $\{\texttt{A,B}\}$,$\texttt{A}$ 标号为 $1$,$\texttt{B}$ 标号为 $2$,那么集合 $\{\texttt{A}\}$ 可以转化为 $(1,0)$。

	\subsection{基因片段}

	基因片段是一个向量。记基因片段组成的集合为 $\mathbb{P}$。

	\subsubsection{配子基因片段}
	
	我们用 $\vec G$ 来表示配子基因片段。

	我们可以将一个具有 $k$ 个基因的配子用一个 $k$ 维有序向量 $\{a_i\}$ 表示,$a_i \in \mathbb{G}$。

	\subsubsection{个体基因片段}

	我们用 $\vec I$ 来表示个体基因片段。

	我们可以将一个具有 $k$ 对等位基因的个体用一个 $k$ 维有序向量 $\{(l_i,r_i)\}$表示,$l_i,r_i \in \mathbb{G}$。

	\subsection{基因片段的运算}

	\subsubsection{加法运算 $+$}

	对于 $L,R \in \mathbb{P}$,而且 $L,R$ 同为配子基因片段或个体基因片段,定义加法运算为两基因片段的 \textbf{有序} 拼接。

	如 $(\texttt{A},\texttt{C}) + (\texttt{B}) = (\texttt{A},\texttt{B},\texttt{C})$。

	\subsubsection{结合运算 $\oplus$}

	对于 $L,R \in \mathbb{P}$,而且 $L,R$ 同为配子基因片段,而且长度相等,定义结合运算为按位有序结合:

	$$(L \oplus R)_i=(\max(L_i,R_i),\min(L_i,R_i))$$

	$\max,\min$ 为取序号较大/较小者。

	如 $(\texttt{A},\texttt{b}) + (\texttt{a},\texttt{B})=(\texttt{(A,a)},\texttt{(B,b)})$。

	\subsection{生成函数 (Generating function)} 
	
	定义:

	$$\textbf{A}=\sum_{i} a_i x^{i}$$

	是序列 $\{a_i\}$ 的生成函数。

	我们不关心 $x$ 的取值和级数是否收敛,把 $x$ 作为形式,只关心系数 $a_i$。

	\subsection{基因片段生成函数}

	定义:

	$$\textbf{A}=\sum_{i \in \mathbb{P}} a_i x^{i}$$

	是序列 $\{a_i\}$ 的基因片段生成函数。

	\subsection{基因片段生成函数的系列运算}
	
	\subsubsection{乘法运算 $\times$}
	
	$$x^L \times x^R=x^{L+R}$$

	\subsubsection{结合乘法运算 $\otimes$}

	$$x^L \otimes x^R=x^{L \oplus R}$$

	\subsection{基因片段生成函数的应用}

	\begin{itemize}
		\item 求基因型为 $\texttt{AaBB}$ 的个体产生的配子数量比。
	\end{itemize}

	构造生成函数:
	
	$$\begin{aligned}
		\textbf{G} &= (\frac{1}{2} x^{\texttt{A}}+ \frac{1}{2}x^{\texttt{a}})(\frac{1}{2} x^{\texttt{B}}+ \frac{1}{2}x^{\texttt{B}}) \\
		           &= \frac{1}{2} x^{\texttt{AB}} + \frac{1}{2} x^{\texttt{aB}}
	\end{aligned}$$

	即配子数量比为 $\texttt{AB} : \texttt{aB}=1:1$。

	\begin{itemize}
		\item 求其自交后个体的基因型比例。
	\end{itemize}

	构造生成函数:

	$$\begin{aligned}
		\textbf{I} &= \textbf{G} \otimes \textbf{G} \\
		           &= \frac{1}{4} x^{\texttt{AABB}} + \frac{1}{2} x^{\texttt{AaBB}} + \frac{1}{4} x^{\texttt{aaBB}}
	\end{aligned}$$

	即基因型数量比为 $\texttt{AABB} : \texttt{AaBB} : \texttt{aaBB}=1:2:1$。

	\subsection{表现型集合}

	定义 $\mathbb{E}$ 为表现型集合,一般地,$\mathbb{E}=\mathbb{G}$。

	\subsection{表现型映射}

	我们创建映射:$exp:\mathbb{G} \times \mathbb{G} \to \mathbb{E}$,对于一对等位基因 $l,r \in G$ 使得 $exp(l,r)$ 为这个个体的表现型。

	比如 $exp(\texttt{A},\texttt{a})=\texttt{A}$,$exp(\texttt{a},\texttt{a})=\texttt{a}$。

	\subsection{表现型映射的性质}

	\begin{itemize}
		\item $exp(i,j)=exp(j,i)$。
		\item $exp(i,i)=i$。
	\end{itemize}

	\subsection{计算个体的表现型}

	个体的表现型可以用一个 $k$ 维向量 $\vec E$ 表示,其中

	$$\vec E_i=exp(\vec I_i)$$

	\subsection{卷积}

	给定环 $R$ 上的 $n$ 维向量 $\vec A=\{a_i\},\vec B=\{b_i\}$ 和下标运算 $\circ$,设 $C=\{c_i\}=A*B$,则满足:

	$$c_i=\sum_{j,k} [j \circ k=i] a_jb_k$$

	称 $C$ 为 $A$ 和 $B$ 关于 $\circ$ 的离散卷积,以下简称卷积。

	记 $C=A*_{\circ}B$,如果不引起混淆,简记为 $C=A*B$,其中 $*$ 为卷积算子。

	若 $\circ = +$,就是我们熟悉的多项式乘法运算。

	\subsection{卷积与生成函数运算的联系}

	若满足运算 $x^L \times x^R = x^{L \circ R}$,那么生成函数 $\mathbf{F}=\sum f_i x^i$ 的乘法:

	$$\textbf{H}=\textbf{F} \times \textbf{G}$$

	和卷积 $\vec F=\{f_i\},\vec G=\{g_i\},\vec H=\vec F *_{\circ} \vec G=\{h_i\}$ 等价。

	\subsection{快速沃尔什变换的实现}

	\begin{algorithm}
        \caption{快速莫比乌斯变换}
        \begin{algorithmic}[1]
            \Require 集合幂级数 $f$
            \Ensure $f$ 的莫比乌斯变换
            \Function {FastMobiusTransform}{$f$}
				\For{$i \gets 1 \textbf{ to } n$}
					\For{$\textbf{all }S \subseteq U \setminus \{i\}$}
						\State $f_{S \cup \{i\}} \gets f_{S \cup \{i\}} + f_{S}$
					\EndFor
				\EndFor
                \State \Return{$f$}
            \EndFunction
		\end{algorithmic}
	\end{algorithm}

	\begin{algorithm}
		\caption{快速莫比乌斯反演}
		\begin{algorithmic}[1]
			\Require 集合幂级数 $f$
			\Ensure $f$ 的莫比乌斯反演
			\Function {FastMobiusInversion}{$f$}
				\For{$i \gets 1 \textbf{ to } n$}
					\For{$\textbf{all }S \subseteq U \setminus \{i\}$}
						\State $f_{S \cup \{i\}} \gets f_{S \cup \{i\}} - f_{S}$
					\EndFor
				\EndFor
				\State \Return{$f$}
			\EndFunction
		\end{algorithmic}
	\end{algorithm}

	\begin{algorithm}
		\caption{多维广义离散傅里叶变换}
		\begin{algorithmic}[1]
			\Require 幂级数 $f$,单位根 $w_k$,操作符 $opr$ 代表正变换还是逆变换。
			\Ensure $f$ 的傅里叶变换
			\Function {FourierTransform}{$f,w_k,opr$}
				\If $\ opr=1$
					\State $matrix_{i,j}=w^{(i-1)(j-1)}_k$
				\Else
					\State $matrix_{i,j}=\frac{1}{k}w^{-(i-1)(j-1)}_k$
				\EndIf
				\For{$i \gets 1 \textbf{ to } n$}
					\For{$\texttt{The k vectors satisfying }1 \cdots k\texttt{ on the i-th bit and the other bits are same.}$}
						\State $v \gets \texttt{the k vectors}$
						\For{$j \gets 1 \textbf{ to } k$}
							\State $g_j \gets f_{v_j}$
						\EndFor
						\State $g \gets g \times matrix$
						\For{$j \gets 1 \textbf{ to } k$}
							\State $f_{v_j} \gets g_j$
						\EndFor
					\EndFor
				\EndFor
				\State \Return{$f$}
			\EndFunction
		\end{algorithmic}
	\end{algorithm}
\end{document}